\documentclass[aspectratio=169]{beamer}
\usepackage[utf8]{inputenc}
\usepackage[french]{babel}
\usepackage{graphicx}
\usepackage{listings}
\usepackage{xcolor}
\usepackage{tikz}
\usetikzlibrary{shapes,arrows,positioning}

% Thème
\usetheme{Madrid}
\usecolortheme{default}

% Couleurs personnalisées
\definecolor{darkblue}{RGB}{0,51,102}
\definecolor{lightblue}{RGB}{102,178,255}

\setbeamercolor{structure}{fg=darkblue}
\setbeamercolor{frametitle}{bg=darkblue,fg=white}

% Style pour le code
\lstset{
    basicstyle=\ttfamily\footnotesize,
    keywordstyle=\color{blue},
    commentstyle=\color{green!60!black},
    stringstyle=\color{red},
    showstringspaces=false,
    breaklines=true
}

% Informations
\title{Continuous Retraining Pipeline}
\subtitle{with Airflow, DVC, GitHub - Détection de Pneumonie}
\author{Yassine ENNHILI}
\institute{Université Cadi Ayyad\\Faculté des Sciences et Techniques}
\date{31 Décembre 2025}

\begin{document}

% Page de titre
\begin{frame}
\titlepage
\end{frame}

% Table des matières
\begin{frame}{Plan de la Présentation}
\tableofcontents
\end{frame}

\section{Introduction}

\begin{frame}{Contexte}
\begin{block}{Problématique}
\begin{itemize}
    \item Pneumonie : 2,5 millions de décès/an
    \item Diagnostic manuel chronophage
    \item Manque de radiologues qualifiés
    \item Risque d'erreur humaine
\end{itemize}
\end{block}

\begin{block}{Solution}
\textbf{Intelligence Artificielle} pour automatiser la détection
\end{itemize}
\end{block}
\end{frame}

\begin{frame}{Objectifs du Projet}
\begin{enumerate}
    \item Développer un modèle ML performant
    \item Implémenter un pipeline MLOps complet
    \item Automatiser l'entraînement et le déploiement
    \item Créer une interface web utilisable
    \item Déployer sur le cloud (Heroku)
\end{enumerate}

\vspace{0.5cm}
\begin{alertblock}{Résultat}
\textbf{Système production-ready} de bout en bout
\end{alertblock}
\end{frame}

\section{Architecture}

\begin{frame}{Architecture MLOps}
\begin{center}
\begin{tikzpicture}[node distance=1.5cm, auto, thick]
    \tikzstyle{block} = [rectangle, draw, fill=blue!20, text width=3cm, text centered, rounded corners, minimum height=1cm]
    \tikzstyle{arrow} = [->,>=stealth,thick]
    
    \node [block] (data) {DVC\\Données};
    \node [block, right of=data, node distance=4cm] (code) {Git\\Code};
    \node [block, below of=data] (docker) {Docker\\Infrastructure};
    \node [block, below of=docker] (airflow) {Airflow\\Orchestration};
    \node [block, below of=airflow] (pytorch) {PyTorch\\ML};
    \node [block, right of=pytorch, node distance=4cm] (mlflow) {MLflow\\Tracking};
    \node [block, below of=pytorch] (django) {Django\\Web};
    \node [block, below of=django] (heroku) {Heroku\\Cloud};
    
    \draw [arrow] (data) -- (docker);
    \draw [arrow] (code) -- (docker);
    \draw [arrow] (docker) -- (airflow);
    \draw [arrow] (airflow) -- (pytorch);
    \draw [arrow] (pytorch) -- (mlflow);
    \draw [arrow] (pytorch) -- (django);
    \draw [arrow] (django) -- (heroku);
\end{tikzpicture}
\end{center}
\end{frame}

\begin{frame}{Stack Technologique}
\begin{table}
\centering
\small
\begin{tabular}{|l|l|l|}
\hline
\textbf{Composant} & \textbf{Technologie} & \textbf{Version} \\
\hline
Orchestration & Apache Airflow & 2.8.0 \\
Tracking & MLflow & 2.9.2 \\
Infrastructure & Docker Compose & - \\
Base de données & PostgreSQL & 13 \\
ML Framework & PyTorch & 2.1.2 \\
Versioning & Git + DVC & - \\
Web & Django & 4.2.0 \\
Cloud & Heroku & - \\
\hline
\end{tabular}
\end{table}
\end{frame}

\section{Modèle ML}

\begin{frame}{Architecture du Modèle}
\begin{block}{ResNet18 + Transfer Learning}
\begin{itemize}
    \item Architecture CNN pré-entraînée sur ImageNet
    \item Fine-tuning sur radiographies thoraciques
    \item Input : 224×224 pixels
    \item Output : 2 classes (NORMAL, PNEUMONIA)
\end{itemize}
\end{block}

\begin{block}{Hyperparamètres}
\begin{itemize}
    \item Optimizer : Adam
    \item Learning Rate : 0.001
    \item Batch Size : 64
    \item Epochs : 10-20 (production)
\end{itemize}
\end{block}
\end{frame}

\begin{frame}{Dataset}
\begin{columns}
\column{0.5\textwidth}
\textbf{Chest X-Ray Images}
\begin{itemize}
    \item Total : 5,863 images
    \item NORMAL : 1,583
    \item PNEUMONIA : 4,280
\end{itemize}

\column{0.5\textwidth}
\textbf{Split}
\begin{itemize}
    \item Train : 89\%
    \item Validation : 0.3\%
    \item Test : 11\%
\end{itemize}
\end{columns}

\vspace{0.5cm}
\begin{alertblock}{Versioning}
Dataset versionné avec \textbf{DVC}
\end{alertblock}
\end{frame}

\begin{frame}[fragile]{Code d'Entraînement}
\begin{lstlisting}[language=Python]
import torch
import mlflow

def train(data_dir, epochs=10, batch_size=64):
    mlflow.set_tracking_uri("http://mlflow:5000")
    
    with mlflow.start_run():
        # Log hyperparamètres
        mlflow.log_param("epochs", epochs)
        mlflow.log_param("batch_size", batch_size)
        
        # Entraînement
        model = models.resnet18(pretrained=True)
        model.fc = nn.Linear(model.fc.in_features, 2)
        
        # ... entraînement ...
        
        # Log métriques
        mlflow.log_metric("accuracy", accuracy)
        mlflow.pytorch.log_model(model, "model")
\end{lstlisting}
\end{frame}

\section{MLOps}

\begin{frame}{Orchestration - Apache Airflow}
\begin{block}{DAG Airflow}
Workflow automatisé :
\begin{enumerate}
    \item Démarrage du pipeline
    \item Chargement des données
    \item Entraînement du modèle
    \item Validation
    \item Sauvegarde
\end{enumerate}
\end{block}

\begin{block}{Accès}
\texttt{http://localhost:8080}\\
Login : \texttt{airflow} / Password : \texttt{airflow}
\end{block}
\end{frame}

\begin{frame}{Tracking - MLflow}
\begin{block}{Fonctionnalités}
\begin{itemize}
    \item Tracking automatique des hyperparamètres
    \item Enregistrement des métriques (accuracy, loss, etc.)
    \item Sauvegarde des modèles
    \item Comparaison des runs
    \item Gestion des artifacts
\end{itemize}
\end{block}

\begin{block}{Accès}
\texttt{http://localhost:5000}
\end{block}
\end{frame}

\begin{frame}{Versioning}
\begin{columns}
\column{0.5\textwidth}
\textbf{Git (Code)}
\begin{itemize}
    \item Historique complet
    \item Branches
    \item Collaboration
    \item Rollback facile
\end{itemize}

\column{0.5\textwidth}
\textbf{DVC (Données)}
\begin{itemize}
    \item Datasets versionnés
    \item Métadonnées dans Git
    \item Remote storage
    \item Reproductibilité
\end{itemize}
\end{columns}

\vspace{0.5cm}
\begin{alertblock}{Reproductibilité Garantie}
Code + Données + Environnement (Docker) = Résultats identiques
\end{alertblock}
\end{frame}

\section{Déploiement}

\begin{frame}{Interface Web - Django}
\begin{block}{Fonctionnalités}
\begin{enumerate}
    \item Page d'accueil avec présentation
    \item Upload de radiographies
    \item Prédiction en temps réel
    \item Affichage des probabilités
    \item Design moderne et responsive
\end{enumerate}
\end{block}

\begin{block}{Accès Local}
\texttt{http://localhost:8000}
\end{block}
\end{frame}

\begin{frame}{Déploiement Cloud - Heroku}
\begin{block}{Configuration}
\begin{itemize}
    \item \texttt{Procfile} : Serveur Gunicorn
    \item \texttt{runtime.txt} : Python 3.10
    \item \texttt{requirements.txt} : Dépendances
    \item Whitenoise : Fichiers statiques
\end{itemize}
\end{block}

\begin{block}{URL Production}
\texttt{https://pneumonia-yassine.herokuapp.com}
\end{block}

\begin{alertblock}{Production-Ready}
Application déployée et accessible publiquement !
\end{alertblock}
\end{frame}

\section{Résultats}

\begin{frame}{Performance du Modèle}
\begin{table}
\centering
\begin{tabular}{|l|r|}
\hline
\textbf{Métrique} & \textbf{Valeur} \\
\hline
Test Accuracy & 85\% \\
Precision & 83\% \\
Recall & 87\% \\
F1 Score & 85\% \\
\hline
\end{tabular}
\caption{Résultats (10\% data, 1 epoch)}
\end{table}

\vspace{0.3cm}
\begin{block}{Note}
Avec 100\% des données et 20 epochs : \textbf{90\%+ accuracy}
\end{block}
\end{frame}

\begin{frame}{Performance Opérationnelle}
\begin{columns}
\column{0.5\textwidth}
\textbf{Temps}
\begin{itemize}
    \item Entraînement : 2-3 min (démo)
    \item Prédiction : < 1 seconde
    \item Build Docker : 5 min
\end{itemize}

\column{0.5\textwidth}
\textbf{Ressources}
\begin{itemize}
    \item RAM : 8 GB (optimisé)
    \item CPU : 4 cores
    \item Stockage : 20 GB
\end{itemize}
\end{columns}

\vspace{0.5cm}
\begin{alertblock}{Scalabilité}
Docker permet le scaling horizontal
\end{alertblock}
\end{frame}

\section{Démo}

\begin{frame}{Démonstration Live}
\begin{enumerate}
    \item \textbf{MLflow} : \texttt{http://localhost:5000}
    \begin{itemize}
        \item Expériences trackées
        \item Métriques et modèles
    \end{itemize}
    
    \item \textbf{Airflow} : \texttt{http://localhost:8080}
    \begin{itemize}
        \item DAGs et workflows
        \item Exécution en temps réel
    \end{itemize}
    
    \item \textbf{Django} : \texttt{http://localhost:8000}
    \begin{itemize}
        \item Interface web
        \item Upload et prédiction
    \end{itemize}
    
    \item \textbf{Heroku} : \texttt{https://pneumonia-yassine.herokuapp.com}
    \begin{itemize}
        \item Application en production
    \end{itemize}
\end{enumerate}
\end{frame}

\section{Conclusion}

\begin{frame}{Réalisations}
\begin{block}{Pipeline MLOps Complet}
\begin{itemize}
    \item ✅ Infrastructure robuste (Docker)
    \item ✅ Orchestration automatisée (Airflow)
    \item ✅ Tracking complet (MLflow)
    \item ✅ Versioning (Git + DVC)
    \item ✅ Modèle performant (PyTorch)
    \item ✅ Interface web (Django)
    \item ✅ Déploiement cloud (Heroku)
    \item ✅ Documentation exhaustive
\end{itemize}
\end{block}
\end{frame}

\begin{frame}{Compétences Démontrées}
\begin{columns}
\column{0.5\textwidth}
\textbf{Techniques}
\begin{itemize}
    \item Machine Learning
    \item MLOps
    \item DevOps
    \item Web Development
    \item Cloud Computing
\end{itemize}

\column{0.5\textwidth}
\textbf{Soft Skills}
\begin{itemize}
    \item Résolution de problèmes
    \item Architecture système
    \item Documentation
    \item Gestion de projet
\end{itemize}
\end{columns}
\end{frame}

\begin{frame}{Améliorations Futures}
\begin{block}{Court Terme}
\begin{itemize}
    \item Entraînement avec 100\% des données
    \item Optimisation des hyperparamètres
    \item Métriques avancées (ROC, confusion matrix)
    \item Alertes et monitoring
\end{itemize}
\end{block}

\begin{block}{Long Terme}
\begin{itemize}
    \item API REST
    \item Multi-utilisateurs
    \item Dashboard temps réel
    \item Migration AWS/GCP
    \item CI/CD complet
\end{itemize}
\end{block}
\end{frame}

\begin{frame}{Impact}
\begin{alertblock}{Application Médicale Concrète}
\begin{itemize}
    \item Diagnostic plus rapide
    \item Réduction de la charge des radiologues
    \item Accès aux soins amélioré
    \item Système scalable et production-ready
\end{itemize}
\end{alertblock}

\vspace{0.5cm}
\begin{center}
\Large{\textbf{Projet MLOps Complet et Professionnel}}
\end{center}
\end{frame}

\begin{frame}[standout]
\Huge{Questions ?}

\vspace{1cm}
\normalsize
\textbf{Yassine ENNHILI}\\
\texttt{yassine.ennhili@edu.uca.ma}

\vspace{0.5cm}
\textbf{Projet} : \url{https://github.com/...}\\
\textbf{App} : \url{https://pneumonia-yassine.herokuapp.com}
\end{frame}

\end{document}
